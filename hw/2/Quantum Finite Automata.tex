\documentclass[letter, 12pt]{article}
\usepackage{comment} % enables the use of multi-line comments (\ifx \fi) 
\usepackage{lipsum} %This package just generates Lorem Ipsum filler text. 
\usepackage{graphicx}
\usepackage{fullpage} % changes the margin
\usepackage{natbib}
\bibliographystyle{abbrvnat}
\setcitestyle{authoryear,open={(},close={)}}

\begin{document}
%Update this information!!!!
\noindent
\large\textbf{CMPT 440 -- Spring 2019: Quantum Finite Automata} \\ \\
\textbf{Christian Santiago} \\
\normalsize   Due Date: 22/4/2019


\section*{Theoretical Background}
Quantum finite automata has been categorized into distinct types of quantum state machines. The  quantum state machine can be defined broadly by the 3-tuple that is $M = (S, s_0,\delta)$. Where $\delta(s,t)$ is the amplitude of the transition function from $s$ to $t$ and probability of the transition is $|\delta(s,t)|^2$. The probability of 1 denotes that the transition is probabilistic and 0 denoting the opposite. The Hilber space is used as a computational basis for $M$ based on the unit circle for calculating probability. The elements of $S$ \emph{states} and $\hat{H}$ as \emph{superposition states}. The unitary operator describes the operation for the superposition of states $\psi = \Sigma \alpha_i s_i$ and the amplitude also defines the probability of the superposition states as $|\langle \psi , s_i \rangle|^2$. The q-\emph{state machine} is then written as $M = (H, s_0, U)$ A quantum state machine of with accepting states is finally written as the following:
\begin{equation}
    M_f = (S, s_0,\delta, S_f)
\end{equation}
Where $M = (S, s_0,\delta)$ is a QSM.
This formula introduced by \cite{Gudder2000}. From this 1-way and 2-way finite automata can be defined. These are further categorized as measure-once quantum finite automata (MO-QFAs) defined as $M = (Q, \Sigma,\delta, q_o, F)$  and measure-many quantum finite automata (MM-QFAs) defined as $M = (Q, \Sigma,\delta, q_o, Q_{acc}, Q_{rej})$. The difference being, measuring after calculation and measuring after each transition receptively. \cite{Brodsky}

\section*{An Example}
The result of these quantum finite automata allow the acceptance of non-regular languages such as $L = a^nb^n$ The two way finite automata can check each condition in each direction not limited to only checking in a single direction. This allows for the start to be in some arbitrary position $s_i$ in the string input and work from that point. This allows a check of both side at a time for which a regular DFA will not be able model without expanding the DFA by the length of the string.

% This is how you cite
%	\cite{Gudder2000}
%	\cite{Brodsky}



%=============================================
\bibliographystyle{acl}
\bibliography{Cites.bib}

\end{document}


\ifx
Comments!
\fi

% ===========

\ifx

%==============

% References if you want it manual

% \bibitem{Robotics} Fred G. Martin \emph{Robotics Explorations: A Hands-On Introduction to Engineering}. New Jersey: Prentice Hall.

% \bibitem{Flueck}  Flueck, Alexander J. 2005. \emph{ECE 100}[online]. Chicago: Illinois Institute of Technology, Electrical and Computer Engineering Department, 2005 [cited 30
% August 2005]. Available from World Wide Web: (http://www.ece.iit.edu/~flueck/ece100).

